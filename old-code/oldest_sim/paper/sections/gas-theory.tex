\documentclass[ ../main.tex]{subfiles}
\providecommand{\mainx}{..}
\begin{document}
\section{Kinetic theory of gas}





Given a particle $r$ moving uniformly at random...
\begin{equation}
    F_x = \frac{N m \bar{v_x^2}}{2 L}
\end{equation}




Partitition universe into a grid. Each grid will deal aggregately and stochastically with mass dispersion.

Mass has two properties in this system: average velocity and variance from average velocity. The variance is a measure of temperature. The average velocity can be low, but the variance can be extremely large such that an magnitude of the velocity of an individual particle is likely to be very large.

Adjacent grids may have different temperatures. Standard calculations may be made to compute the heat transfer between these adjacent regions. In addition, since there is an average velocity to these masses, a proportion of the mass in a grid will depart. 


\end{document}